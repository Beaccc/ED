\input{preambuloSimple.tex}

%----------------------------------------------------------------------------------------
%	TÍTULO Y DATOS DEL ALUMNO
%----------------------------------------------------------------------------------------

\title{	
\normalfont \normalsize 
\textsc{\textbf{Estructura de Datos (2016-2017)} \\ Subgrupo C2 \\ Grado en Ingeniería Informática\\ Universidad de Granada} \\ [25pt] % Your university, school and/or department name(s)
\horrule{0.5pt} \\[0.4cm] % Thin top horizontal rule
\huge Relación 1: Eficiencia \\ % The assignment title
\horrule{2pt} \\[0.5cm] % Thick bottom horizontal rule
}

\author{Mario Rodríguez Ruiz} % Nombre y apellidos

\date{\normalsize\today} % Incluye la fecha actual

%----------------------------------------------------------------------------------------
% DOCUMENTO
%----------------------------------------------------------------------------------------

\begin{document}

\maketitle % Muestra el Título

\newpage %inserta un salto de página

\tableofcontents % para generar el índice de contenidos

\newpage

%----------------------------------------------------------------------------------------
%	Cuestión 1.3
%----------------------------------------------------------------------------------------
\section{Problema 1.3}
\textbf{\textit{Ordenar de menor a mayor los siguientes órdenes de eficiencia:}}
\begin{gather*}
	n, \sqrt{n}, n^3+1, n^2, nlog_{2}(n^2),nlog_{2}log_{2}(n^2), 3^{log_{2}(n)}, 3^n, 2^n, 2^n+3^{n-1}, 20000, n + 100, n2^n\\
	nlog_{2}log_{2}(n^2) <  nlog_{2}(n^2) < 3^{log_{2}(n)} < \sqrt{n} < n < n^2 < 2^n < 2^n + 3^{n-1} < \\
	< n2^n < n^3 + 1 < 3^n < n + 100 < 20000
\end{gather*}

%----------------------------------------------------------------------------------------
%	Cuestión 1.9
%----------------------------------------------------------------------------------------
\section{Problema 1.6}
\textbf{\textit{Demostrar que si f(n) $ \ni $ O(g(n)) y g(n) $ \ni $ O(h(n)) entonces f(n) $ \ni $ O(h(n))}}
\begin{gather*}
	O(n) = \{1, 10, 5, log_{2}(n), log_{7}(n),3 log_{2}(n), \sqrt{n}, n, ...\}\\	
	O(n^2) = \{n^2,..., 3n^2+n,...\} U O(n) \\
	f(n) = log_{2}(n) \ni O(log_{2}(n))\\
	f(n) = log_{2}(n) \ni O(n)\\
	g(n) = n^2 \ni O(n^2)
\end{gather*}


%----------------------------------------------------------------------------------------
%	Cuestión 1.9
%----------------------------------------------------------------------------------------
\section{Problema 1.9}
\textbf{\textit{Obtener usando la notación O-mayúscula la eficiencia del siguiente trozo de código:}}
\begin {lstlisting}						
for (int i=0; i<n; i++)						
	for (int j=0; j<n; j++){
		C[i][j]= 0;
		for (k=0;k<n;k++)
			C[i][j]+=A[i][k]*B[k][j];
	}
\end{lstlisting}

\begin{itemize}
\item Linea 1: 3 OE (asignación, comparación, incremento) 
\item Linea 2: 3 OE (asignación, comparación, incremento)
\item Linea 3: 2 OE (acceso al elemento C[[i][j], asignación)
\item Linea 4: 3 OE (asignación, comparación, incremento)
\item Linea 5: 7 OE (4 accesos a elementos de las matrices, 2 operaciones, asignación)
\end{itemize}
Se tienen tres bucles anidados, haciendo cada uno \textbf{n iteraciones}. El resto de la función es de orden constante, por lo que puede identificarse cada una como \textbf{O(1)}. De esta forma, el tiempo de ejecución en función de n es:

\begin{equation}
	T(n)=\sum_{i=0}^{n}\sum_{j=0}^{n}\sum_{k=0}^{n}1= \sum_{i=0}^{n}\sum_{j=0}^{n}n=n^3
\end{equation}

\begin{equation}
\textup{Por lo que se puede confirmar que la eficiencia del código es de } O(n^3)
\end{equation}


%----------------------------------------------------------------------------------------
%	Cuestión 1.10
%----------------------------------------------------------------------------------------
\section{Problema 1.10}
\textbf{\textit{Obtener usando la notación O-mayúscula la eficiencia de la siguiente función:}}
\begin {lstlisting}						
void ejemplo(int n)
{
	int i, j, k;
	for (i = 1; i < n; i++)
		for (j = i+1; j <= n; j++)
			for (k = 1; k <= j; k++)
				Global += k*i;
}
\end{lstlisting}

\begin{itemize}
	\item Linea 4: 3 OE (asignación, comparación, incremento) 
	\item Linea 5: 4 OE (asignación, comparación, 2 incrementos)
	\item Linea 6: 4 OE (asignación, comparación, incremento)
	\item Linea 7: 3 OE (asignación, 2 operaciones)
\end{itemize}
Bucle de la j
\begin{equation}
	Tj (n)= \frac{(n-i)*(1+n-i)}{2} \rightarrow O(n^2)
\end{equation}

Bucle de la i
\begin{equation}
	Ti (n)= \sum_{i=0}^{n}(\frac{(n-i)*(1+n-i)}{2}) 	
\end{equation}
\begin{equation}
	\textup{Por lo que el orden de eficiencia de la función es de }  O(n^2)
\end{equation}

\newpage

%----------------------------------------------------------------------------------------
%	Cuestión 1.11
%----------------------------------------------------------------------------------------
\section{Problema 1.11}
\textbf{\textit{Obtener usando la notación O-mayúscula la eficiencia del siguiente trozo de código:}}
\begin {lstlisting}						
	for (i=0;i<n;i++)
		if (i%2){
			for (j=i;j<n;j++)
				x*=i ;
			for (j=1;j<i;j++)
				y*=j ;
		}
\end{lstlisting}

\begin{itemize}
	\item Linea 1: 3 OE (asignación, comparación, incremento) 
	\item Linea 2: 1 OE (operación)
	\item Linea 3: 3 OE (asignación, comparación, incremento)
	\item Linea 4: 2 OE (producto, asignación)
	\item Linea 5: 3 OE (asignación, comparación, incremento)
	\item Linea 6: 2 OE (producto, asignación)
\end{itemize}

Bucle de la j
\begin{equation}
\sum_{j=i}^{n}1 = n-i 
\end{equation}
\begin{equation}
\sum_{j=1}^{i}1 = i 
\end{equation}
\begin{equation}
Tj (n) = n - i + i = n  
\end{equation}

\begin{equation}
T(n)=\sum_{i=0}^{n}n = n\sum_{i=0}^{\frac{n}{2}}1 = n \frac{n}{2} = \frac{n^2}{2}
\end{equation}

\begin{equation}
\textup{Por lo que se puede confirmar que la eficiencia del código es de } O(n^2)
\end{equation}

\newpage
\end{document}
