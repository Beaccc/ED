%%%%%%%%%%%%%%%%%%%%%%%%%%%%%%%%%%%%%%%%%
% Short Sectioned Assignment LaTeX Template Version 1.0 (5/5/12)
% This template has been downloaded from: http://www.LaTeXTemplates.com
% Original author:  Frits Wenneker (http://www.howtotex.com)
% License: CC BY-NC-SA 3.0 (http://creativecommons.org/licenses/by-nc-sa/3.0/)
%%%%%%%%%%%%%%%%%%%%%%%%%%%%%%%%%%%%%%%%%

%----------------------------------------------------------------------------------------
%	PACKAGES AND OTHER DOCUMENT CONFIGURATIONS
%----------------------------------------------------------------------------------------

\documentclass[paper=a4, fontsize=11pt]{scrartcl} % A4 paper and 11pt font size

% ---- Entrada y salida de texto -----

\usepackage[T1]{fontenc} % Use 8-bit encoding that has 256 glyphs
\usepackage[utf8]{inputenc}
%\usepackage{fourier} % Use the Adobe Utopia font for the document - comment this line to return to the LaTeX default

% ---- Idioma --------

\usepackage[spanish, es-tabla]{babel} % Selecciona el español para palabras introducidas automáticamente, p.ej. "septiembre" en la fecha y especifica que se use la palabra Tabla en vez de Cuadro

% ---- Otros paquetes ----

\usepackage{url} % ,href} %para incluir URLs e hipervínculos dentro del texto (aunque hay que instalar href)
\usepackage{amsmath,amsfonts,amsthm} % Math packages
%\usepackage{graphics,graphicx, floatrow} %para incluir imágenes y notas en las imágenes
\usepackage{graphics,graphicx, float} %para incluir imágenes y colocarlas

% Para hacer tablas comlejas
%\usepackage{multirow}
%\usepackage{threeparttable}

%\usepackage{sectsty} % Allows customizing section commands
%\allsectionsfont{\centering \normalfont\scshape} % Make all sections centered, the default font and small caps

\usepackage{fancyhdr} % Custom headers and footers

\pagestyle{fancyplain} % Makes all pages in the document conform to the custom headers and footers
\fancyhead{} % No page header - if you want one, create it in the same way as the footers below
\fancyfoot[L]{} % Empty left footer
\fancyfoot[C]{} % Empty center footer
\fancyfoot[R]{\thepage} % Page numbering for right footer
\renewcommand{\headrulewidth}{0pt} % Remove header underlines
\renewcommand{\footrulewidth}{0pt} % Remove footer underlines
\setlength{\headheight}{13.6pt} % Customize the height of the header

\numberwithin{equation}{section} % Number equations within sections (i.e. 1.1, 1.2, 2.1, 2.2 instead of 1, 2, 3, 4)
\numberwithin{figure}{section} % Number figures within sections (i.e. 1.1, 1.2, 2.1, 2.2 instead of 1, 2, 3, 4)
\numberwithin{table}{section} % Number tables within sections (i.e. 1.1, 1.2, 2.1, 2.2 instead of 1, 2, 3, 4)

\setlength\parindent{0pt} % Removes all indentation from paragraphs - comment this line for an assignment with lots of text

\newcommand{\horrule}[1]{\rule{\linewidth}{#1}} % Create horizontal rule command with 1 argument of height
\usepackage[breaklinks=true]{hyperref}

\usepackage[dvipsnames]{xcolor}
\usepackage{amssymb}
\usepackage{color}
\usepackage{listings}
\usepackage{upgreek} % para poner letras griegas sin cursiva
\usepackage{cancel} % para tachar
\usepackage{mathdots} % para el comando \iddots
\usepackage{mathrsfs} % para formato de letra
\usepackage{stackrel} % para el comando \stackbin
\lstset{ %
language=C++,                % elegir el lenguaje del código
stringstyle=\color{blue}\ttfamily,,
basicstyle=\normalsize\ttfamily,       % el tamaño del font a usar para el código
numbers=left,                   % dónde poner los números de línea 
numberstyle=\footnotesize,      % tamaño de font usados para los números de línea 
stepnumber=1,                   % el paso de numeración
numbersep=5pt,                  % distancia del numero de línea y la línea
backgroundcolor=\color{white},  % color de fondo, para usarlo hay que agregar  \usepackage{color}
showspaces=false,               % mostrar espacios en blanco ?
showstringspaces=false,         % subrayar espacios con cadenas?   
 showtabs=false,                 % mostrar taba usando cadenas? 
frame=single,           			% enmarcar el código?  
tabsize=2,          				% sets default tabsize to 2 spaces?
keywordstyle=\color{MidnightBlue}\ttfamily\bfseries,
commentstyle=\color{OliveGreen}\ttfamily,
morecomment=[l][\color{OliveGreen}]{\#},
captionpos=b,           % sets the caption-position to bottom?
breaklines=true,        % sets automatic line breaking?
breakatwhitespace=false,    % sets if automatic breaks should only happen at whitespace ?
title=\lstname,
escapeinside={\%*}{*)}          % if you want to add a comment within your code
}

\lstset{literate=
  {á}{{\'a}}1 {é}{{\'e}}1 {í}{{\'i}}1 {ó}{{\'o}}1 {ú}{{\'u}}1
  {Á}{{\'A}}1 {É}{{\'E}}1 {Í}{{\'I}}1 {Ó}{{\'O}}1 {Ú}{{\'U}}1
  {à}{{\`a}}1 {è}{{\`e}}1 {ì}{{\`i}}1 {ò}{{\`o}}1 {ù}{{\`u}}1
  {À}{{\`A}}1 {È}{{\'E}}1 {Ì}{{\`I}}1 {Ò}{{\`O}}1 {Ù}{{\`U}}1
  {ä}{{\"a}}1 {ë}{{\"e}}1 {ï}{{\"i}}1 {ö}{{\"o}}1 {ü}{{\"u}}1
  {Ä}{{\"A}}1 {Ë}{{\"E}}1 {Ï}{{\"I}}1 {Ö}{{\"O}}1 {Ü}{{\"U}}1
  {â}{{\^a}}1 {ê}{{\^e}}1 {î}{{\^i}}1 {ô}{{\^o}}1 {û}{{\^u}}1
  {Â}{{\^A}}1 {Ê}{{\^E}}1 {Î}{{\^I}}1 {Ô}{{\^O}}1 {Û}{{\^U}}1
  {œ}{{\oe}}1 {Œ}{{\OE}}1 {æ}{{\ae}}1 {Æ}{{\AE}}1 {ß}{{\ss}}1
  {ű}{{\H{u}}}1 {Ű}{{\H{U}}}1 {ő}{{\H{o}}}1 {Ő}{{\H{O}}}1
  {ç}{{\c c}}1 {Ç}{{\c C}}1 {ø}{{\o}}1 {å}{{\r a}}1 {Å}{{\r A}}1
  {€}{{\EUR}}1 {£}{{\pounds}}1
  {ñ}{{\~n}}1
}



\hypersetup{
    colorlinks=true,
    linkcolor=black,
    filecolor=magenta,      
    urlcolor=blue,
    pdftitle={ED: Práctica 3 - Mario Rodríguez Ruiz},
    bookmarks=true,
}



%----------------------------------------------------------------------------------------
%	TÍTULO Y DATOS DEL ALUMNO
%----------------------------------------------------------------------------------------

\title{	
\normalfont \normalsize 
\textsc{\textbf{Estructura de Datos (2016-2017)} \\ Subgrupo C2 \\ Grado en Ingeniería Informática\\ Universidad de Granada} \\ [25pt] % Your university, school and/or department name(s)
\horrule{0.5pt} \\[0.4cm] % Thin top horizontal rule
\huge Relación 1: Eficiencia \\ % The assignment title
\horrule{2pt} \\[0.5cm] % Thick bottom horizontal rule
}

\author{Mario Rodríguez Ruiz} % Nombre y apellidos

\date{\normalsize\today} % Incluye la fecha actual

%----------------------------------------------------------------------------------------
% DOCUMENTO
%----------------------------------------------------------------------------------------

\begin{document}

\maketitle % Muestra el Título

\newpage %inserta un salto de página

\tableofcontents % para generar el índice de contenidos

\newpage

%----------------------------------------------------------------------------------------
%	Cuestión 1.3
%----------------------------------------------------------------------------------------
\section{Problema 1.3}
\textbf{\textit{Ordenar de menor a mayor los siguientes órdenes de eficiencia:}}
\begin{gather*}
	n, \sqrt{n}, n^3+1, n^2, nlog_{2}(n^2),nlog_{2}log_{2}(n^2), 3^{log_{2}(n)}, 3^n, 2^n, 2^n+3^{n-1}, 20000, n + 100, n2^n\\
	nlog_{2}log_{2}(n^2) <  nlog_{2}(n^2) < 3^{log_{2}(n)} < \sqrt{n} < n < n^2 < 2^n < 2^n + 3^{n-1} < \\
	< n2^n < n^3 + 1 < 3^n < n + 100 < 20000
\end{gather*}

%----------------------------------------------------------------------------------------
%	Cuestión 1.9
%----------------------------------------------------------------------------------------
\section{Problema 1.6}
\textbf{\textit{Demostrar que si f(n) $ \ni $ O(g(n)) y g(n) $ \ni $ O(h(n)) entonces f(n) $ \ni $ O(h(n))}}
\begin{gather*}
	O(n) = \{1, 10, 5, log_{2}(n), log_{7}(n),3 log_{2}(n), \sqrt{n}, n, ...\}\\	
	O(n^2) = \{n^2,..., 3n^2+n,...\} U O(n) \\
	f(n) = log_{2}(n) \ni O(log_{2}(n))\\
	f(n) = log_{2}(n) \ni O(n)\\
	g(n) = n^2 \ni O(n^2)
\end{gather*}


%----------------------------------------------------------------------------------------
%	Cuestión 1.9
%----------------------------------------------------------------------------------------
\section{Problema 1.9}
\textbf{\textit{Obtener usando la notación O-mayúscula la eficiencia del siguiente trozo de código:}}
\begin {lstlisting}						
for (int i=0; i<n; i++)						
	for (int j=0; j<n; j++){
		C[i][j]= 0;
		for (k=0;k<n;k++)
			C[i][j]+=A[i][k]*B[k][j];
	}
\end{lstlisting}

\begin{itemize}
\item Linea 1: 3 OE (asignación, comparación, incremento) 
\item Linea 2: 3 OE (asignación, comparación, incremento)
\item Linea 3: 2 OE (acceso al elemento C[[i][j], asignación)
\item Linea 4: 3 OE (asignación, comparación, incremento)
\item Linea 5: 7 OE (4 accesos a elementos de las matrices, 2 operaciones, asignación)
\end{itemize}
Se tienen tres bucles anidados, haciendo cada uno \textbf{n iteraciones}. El resto de la función es de orden constante, por lo que puede identificarse cada una como \textbf{O(1)}. De esta forma, el tiempo de ejecución en función de n es:

\begin{equation}
	T(n)=\sum_{i=0}^{n}\sum_{j=0}^{n}\sum_{k=0}^{n}1= \sum_{i=0}^{n}\sum_{j=0}^{n}n=n^3
\end{equation}

\begin{equation}
\textup{Por lo que se puede confirmar que la eficiencia del código es de } O(n^3)
\end{equation}


%----------------------------------------------------------------------------------------
%	Cuestión 1.10
%----------------------------------------------------------------------------------------
\section{Problema 1.10}
\textbf{\textit{Obtener usando la notación O-mayúscula la eficiencia de la siguiente función:}}
\begin {lstlisting}						
void ejemplo(int n)
{
	int i, j, k;
	for (i = 1; i < n; i++)
		for (j = i+1; j <= n; j++)
			for (k = 1; k <= j; k++)
				Global += k*i;
}
\end{lstlisting}

\begin{itemize}
	\item Linea 4: 3 OE (asignación, comparación, incremento) 
	\item Linea 5: 4 OE (asignación, comparación, 2 incrementos)
	\item Linea 6: 4 OE (asignación, comparación, incremento)
	\item Linea 7: 3 OE (asignación, 2 operaciones)
\end{itemize}
Bucle de la j
\begin{equation}
	Tj (n)= \frac{(n-i)*(1+n-i)}{2} \rightarrow O(n^2)
\end{equation}

Bucle de la i
\begin{equation}
	Ti (n)= \sum_{i=0}^{n}(\frac{(n-i)*(1+n-i)}{2}) 	
\end{equation}
\begin{equation}
	\textup{Por lo que el orden de eficiencia de la función es de }  O(n^2)
\end{equation}

\newpage

%----------------------------------------------------------------------------------------
%	Cuestión 1.11
%----------------------------------------------------------------------------------------
\section{Problema 1.11}
\textbf{\textit{Obtener usando la notación O-mayúscula la eficiencia del siguiente trozo de código:}}
\begin {lstlisting}						
	for (i=0;i<n;i++)
		if (i%2){
			for (j=i;j<n;j++)
				x*=i ;
			for (j=1;j<i;j++)
				y*=j ;
		}
\end{lstlisting}

\begin{itemize}
	\item Linea 1: 3 OE (asignación, comparación, incremento) 
	\item Linea 2: 1 OE (operación)
	\item Linea 3: 3 OE (asignación, comparación, incremento)
	\item Linea 4: 2 OE (producto, asignación)
	\item Linea 5: 3 OE (asignación, comparación, incremento)
	\item Linea 6: 2 OE (producto, asignación)
\end{itemize}

Bucle de la j
\begin{equation}
\sum_{j=i}^{n}1 = n-i 
\end{equation}
\begin{equation}
\sum_{j=1}^{i}1 = i 
\end{equation}
\begin{equation}
Tj (n) = n - i + i = n  
\end{equation}

\begin{equation}
T(n)=\sum_{i=0}^{n}n = n\sum_{i=0}^{\frac{n}{2}}1 = n \frac{n}{2} = \frac{n^2}{2}
\end{equation}

\begin{equation}
\textup{Por lo que se puede confirmar que la eficiencia del código es de } O(n^2)
\end{equation}

\newpage
\end{document}
