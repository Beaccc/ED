%%%%%%%%%%%%%%%%%%%%%%%%%%%%%%%%%%%%%%%%%
% Short Sectioned Assignment LaTeX Template Version 1.0 (5/5/12)
% This template has been downloaded from: http://www.LaTeXTemplates.com
% Original author:  Frits Wenneker (http://www.howtotex.com)
% License: CC BY-NC-SA 3.0 (http://creativecommons.org/licenses/by-nc-sa/3.0/)
%%%%%%%%%%%%%%%%%%%%%%%%%%%%%%%%%%%%%%%%%

%----------------------------------------------------------------------------------------
%	PACKAGES AND OTHER DOCUMENT CONFIGURATIONS
%----------------------------------------------------------------------------------------

\documentclass[paper=a4, fontsize=11pt]{scrartcl} % A4 paper and 11pt font size

% ---- Entrada y salida de texto -----

\usepackage[T1]{fontenc} % Use 8-bit encoding that has 256 glyphs
\usepackage[utf8]{inputenc}
%\usepackage{fourier} % Use the Adobe Utopia font for the document - comment this line to return to the LaTeX default

% ---- Idioma --------

\usepackage[spanish, es-tabla]{babel} % Selecciona el español para palabras introducidas automáticamente, p.ej. "septiembre" en la fecha y especifica que se use la palabra Tabla en vez de Cuadro

% ---- Otros paquetes ----

\usepackage{url} % ,href} %para incluir URLs e hipervínculos dentro del texto (aunque hay que instalar href)
\usepackage{amsmath,amsfonts,amsthm} % Math packages
%\usepackage{graphics,graphicx, floatrow} %para incluir imágenes y notas en las imágenes
\usepackage{graphics,graphicx, float} %para incluir imágenes y colocarlas

% Para hacer tablas comlejas
%\usepackage{multirow}
%\usepackage{threeparttable}

%\usepackage{sectsty} % Allows customizing section commands
%\allsectionsfont{\centering \normalfont\scshape} % Make all sections centered, the default font and small caps

\usepackage{fancyhdr} % Custom headers and footers

\pagestyle{fancyplain} % Makes all pages in the document conform to the custom headers and footers
\fancyhead{} % No page header - if you want one, create it in the same way as the footers below
\fancyfoot[L]{} % Empty left footer
\fancyfoot[C]{} % Empty center footer
\fancyfoot[R]{\thepage} % Page numbering for right footer
\renewcommand{\headrulewidth}{0pt} % Remove header underlines
\renewcommand{\footrulewidth}{0pt} % Remove footer underlines
\setlength{\headheight}{13.6pt} % Customize the height of the header

\numberwithin{equation}{section} % Number equations within sections (i.e. 1.1, 1.2, 2.1, 2.2 instead of 1, 2, 3, 4)
\numberwithin{figure}{section} % Number figures within sections (i.e. 1.1, 1.2, 2.1, 2.2 instead of 1, 2, 3, 4)
\numberwithin{table}{section} % Number tables within sections (i.e. 1.1, 1.2, 2.1, 2.2 instead of 1, 2, 3, 4)

\setlength\parindent{0pt} % Removes all indentation from paragraphs - comment this line for an assignment with lots of text

\newcommand{\horrule}[1]{\rule{\linewidth}{#1}} % Create horizontal rule command with 1 argument of height
\usepackage[breaklinks=true]{hyperref}

\usepackage[dvipsnames]{xcolor}
\usepackage{amssymb}
\usepackage{color}
\usepackage{listings}
\usepackage{upgreek} % para poner letras griegas sin cursiva
\usepackage{cancel} % para tachar
\usepackage{mathdots} % para el comando \iddots
\usepackage{mathrsfs} % para formato de letra
\usepackage{stackrel} % para el comando \stackbin
\lstset{ %
language=C++,                % elegir el lenguaje del código
stringstyle=\color{blue}\ttfamily,,
basicstyle=\normalsize\ttfamily,       % el tamaño del font a usar para el código
numbers=left,                   % dónde poner los números de línea 
numberstyle=\footnotesize,      % tamaño de font usados para los números de línea 
stepnumber=1,                   % el paso de numeración
numbersep=5pt,                  % distancia del numero de línea y la línea
backgroundcolor=\color{white},  % color de fondo, para usarlo hay que agregar  \usepackage{color}
showspaces=false,               % mostrar espacios en blanco ?
showstringspaces=false,         % subrayar espacios con cadenas?   
 showtabs=false,                 % mostrar taba usando cadenas? 
frame=single,           			% enmarcar el código?  
tabsize=2,          				% sets default tabsize to 2 spaces?
keywordstyle=\color{MidnightBlue}\ttfamily\bfseries,
commentstyle=\color{OliveGreen}\ttfamily,
morecomment=[l][\color{OliveGreen}]{\#},
captionpos=b,           % sets the caption-position to bottom?
breaklines=true,        % sets automatic line breaking?
breakatwhitespace=false,    % sets if automatic breaks should only happen at whitespace ?
title=\lstname,
escapeinside={\%*}{*)}          % if you want to add a comment within your code
}

\lstset{literate=
  {á}{{\'a}}1 {é}{{\'e}}1 {í}{{\'i}}1 {ó}{{\'o}}1 {ú}{{\'u}}1
  {Á}{{\'A}}1 {É}{{\'E}}1 {Í}{{\'I}}1 {Ó}{{\'O}}1 {Ú}{{\'U}}1
  {à}{{\`a}}1 {è}{{\`e}}1 {ì}{{\`i}}1 {ò}{{\`o}}1 {ù}{{\`u}}1
  {À}{{\`A}}1 {È}{{\'E}}1 {Ì}{{\`I}}1 {Ò}{{\`O}}1 {Ù}{{\`U}}1
  {ä}{{\"a}}1 {ë}{{\"e}}1 {ï}{{\"i}}1 {ö}{{\"o}}1 {ü}{{\"u}}1
  {Ä}{{\"A}}1 {Ë}{{\"E}}1 {Ï}{{\"I}}1 {Ö}{{\"O}}1 {Ü}{{\"U}}1
  {â}{{\^a}}1 {ê}{{\^e}}1 {î}{{\^i}}1 {ô}{{\^o}}1 {û}{{\^u}}1
  {Â}{{\^A}}1 {Ê}{{\^E}}1 {Î}{{\^I}}1 {Ô}{{\^O}}1 {Û}{{\^U}}1
  {œ}{{\oe}}1 {Œ}{{\OE}}1 {æ}{{\ae}}1 {Æ}{{\AE}}1 {ß}{{\ss}}1
  {ű}{{\H{u}}}1 {Ű}{{\H{U}}}1 {ő}{{\H{o}}}1 {Ő}{{\H{O}}}1
  {ç}{{\c c}}1 {Ç}{{\c C}}1 {ø}{{\o}}1 {å}{{\r a}}1 {Å}{{\r A}}1
  {€}{{\EUR}}1 {£}{{\pounds}}1
  {ñ}{{\~n}}1
}



\hypersetup{
    colorlinks=true,
    linkcolor=black,
    filecolor=magenta,      
    urlcolor=blue,
    pdftitle={ED: Práctica 3 - Mario Rodríguez Ruiz},
    bookmarks=true,
}



%----------------------------------------------------------------------------------------
%	TÍTULO Y DATOS DEL ALUMNO
%----------------------------------------------------------------------------------------

\title{	
\normalfont \normalsize 
\textsc{\textbf{Estructura de Datos (2016-2017)} \\ Grupo C \\ Grado en Ingeniería Informática\\ Universidad de Granada} \\ [25pt] % Your university, school and/or department name(s)
\horrule{0.5pt} \\[0.4cm] % Thin top horizontal rule
\huge Relación 2: Abstracción \\ % The assignment title
\horrule{2pt} \\[0.5cm] % Thick bottom horizontal rule
}

\author{Mario Rodríguez Ruiz} % Nombre y apellidos

\date{\normalsize\today} % Incluye la fecha actual

%----------------------------------------------------------------------------------------
% DOCUMENTO
%----------------------------------------------------------------------------------------

\begin{document}

\maketitle % Muestra el Título

\newpage %inserta un salto de página

\tableofcontents % para generar el índice de contenidos

\newpage

%----------------------------------------------------------------------------------------
%	Problema 1
%----------------------------------------------------------------------------------------
\section{Definir el T.D.A Servidor de red. Un servidor es un punto de red que se encuentra
	identificado por una dirección ip. Una dirección ip viene definida por cuatro dígitos que
	pueden tener valores que van desde 0 a 255.}
\subsection{Dar la especificación del tipo Servidor. Además establecer las operaciones que
	manejan al T.D.A}
T.D.A \textbf{Servidor de red}\\

\textbf{Especificación:} 
Representa la dirección IP de un servidor de red definida en 4 dígitos.

\subsubsection{Operaciones:}
\begin{itemize}
	\item Constructores: constructor por defecto, constructor con una IP determinada.
	\item Consulta: acceder a cualquiera de los cuatro dígitos.
	\item Modificadores: de los dígitos.
	\item Escritura y Lectura de una IP por un flujo de entrada y salida, respectivamente.
\end{itemize}

\subsection{Definir al menos dos tipo rep}
\begin{enumerate}[a)]
\item Posibilidad 1:
\begin{lstlisting}[style=cmas]
class ServidorRed{
	private:
		int d1, d2, d3, d4 ;		
\end{lstlisting}
\setlength{\parskip}{-4mm}
\item Posibilidad 2:
\begin{lstlisting}[style=cmas]
class ServidorRed{
	private:
		string direccion ;		
\end{lstlisting}
\setlength{\parskip}{-4mm}
\item Posibilidad 3:
\begin{lstlisting}[style=cmas]
class ServidorRed{
	private:
		int digito[4] ;		
\end{lstlisting}
\end{enumerate}

\subsection{Escoger uno de los tipo rep y para éste establecer la función de abstracción}
Tras haber elegido la posibilidad 3, la \textbf{función de abstracción} se define como:\\
\begin{center}
	$ f_{A}(r) = $ $ " $r.digito[0] . r.digito[1] . r.digito[2] . r.digito[3]$ " $
\end{center}
Donde r es una instancia del objeto abstracto de tipo rep que sirve para representar el T.D.A. Servidor de red

\subsection{Invariante de la representación}
\begin{center}
	$ r.digito[i] \in ['0' - '255']\forall i = 0...3 $ (condición de que cada dígito debe encontrarse en el rango de valores especificado)
\end{center}

\newpage

%----------------------------------------------------------------------------------------
%	Problema 2
%----------------------------------------------------------------------------------------
\section{Definir el T.D.A Subred. Este T.D.A es una colección de servidores (s1,s2,...,sn). En éste también se almacena
	si dos servidores están conectados entre ellos}
\subsection{Dar la especificación del tipo Subred. Además establecer las operaciones que
	manejan a T.D.A}
T.D.A \textbf{Subred}

\textbf{Especificación:} colección de servidores de red (s1,s2,...,sn) que almacena además si existe un enlace directo entre dos servidores.

\subsubsection{Operaciones:}
\begin{itemize}
	\item Constructores: constructor por defecto, constructor con una colección determinada.
	\item Consulta: acceder a cualquiera de los servidores de la colección y al número total de éstos, así como a qué otros servidores se encuentran conectados.
	\item Modificadores: de las conexiones entre servidores.
	\item Escritura y Lectura de una Subred por un flujo de entrada y salida, respectivamente.
\end{itemize}

\subsection{Definir al menos dos tipo rep}
\begin{enumerate}[a)]
\item Posibilidad 1:
\begin{lstlisting}[style=cmas]
struct enlace{
	unsigned *servidores_conectados ; 

class Subred{
	private:
		ServidorRed *s ;
		vector<enlace> enlaces ;
		unsigned num_servers ;			
\end{lstlisting}
\setlength{\parskip}{-4mm}
\item Posibilidad 2:
\begin{lstlisting}[style=cmas]
struct enlace{
	bool *conexiones ;
	
class Subred{
	private:
		ServidorRed *s ;
		vector<enlace> enlaces ;
		unsigned num_servers ;		
\end{lstlisting}
\end{enumerate}

\subsection{Escoger uno de los tipo rep y para éste establecer la función de abstracción}
Tras haber elegido la posibilidad 1, la \textbf{función de abstracción} se define como:\\
\begin{center}
	$ f_{A}(r) = $ $ " $r.s[0], r.s[1],..., r.s[r.num\_servers - 1]$ " $
\end{center}
Donde r es una instancia del objeto abstracto de tipo rep que sirve para representar el T.D.A. Subred

\subsection{Invariante de la representación}
\begin{center}
	$ (r.s[i] \neq r.s[j])\forall i,j $ $0 \leq i < j < r.num\_servers $
\end{center}

\newpage

%----------------------------------------------------------------------------------------
%	Problema 3
%----------------------------------------------------------------------------------------
\section{Definir el T.D.A Punto Geográfico. Un punto geográfico se define por una latitud y longitud.
	La latitud es la distancia en grados desde la línea del Ecuador a los Polos. Su rango va
	desde -90º a 90º. La longitud es la distancia desde el meriadiano 0 al punto donde estamos.
	El rango de valores que adopta va desde -180º a 180º.}
\subsection{Dar la especificación del tipo Punto Geográfico. Además establecer las operaciones
	que manejan a T.D.A}
T.D.A \textbf{Punto Geográfico}\\

\textbf{Especificación:} representa un punto geográfico de tal forma que contiene tanto una latitud (distancia en grados desde la línea del Ecuador a los Polos) como una longitud (distancia desde el meriadiano 0 al punto donde estamos) meridional.

\subsubsection{Operaciones:}
\begin{itemize}
	\item Constructores: constructor por defecto, constructor con una latitud y longitud meridional determinadas.
	\item Consulta: acceder a la latitud y a la longitud.
	\item Modificadores: de la latitud y de la longitud.
	\item Escritura y Lectura de un punto geográfico por un flujo de entrada y salida, respectivamente.
\end{itemize}

\subsection{Definir al menos dos tipo rep}
\begin{enumerate}[a)]
\item Posibilidad 1:
\begin{lstlisting}[style=cmas]
struct puntoGeografico{
	int longitud ; 
	int latitud ; 
\end{lstlisting}
\setlength{\parskip}{-4mm}
\item Posibilidad 2:
\begin{lstlisting}[style=cmas]
class PuntoGeografico{
private:
	int punto[2] ; // punto[0] = longitud ; punto[1] = latitud ;
\end{lstlisting}
\newpage
\item Posibilidad 3:
\begin{lstlisting}[style=cmas]
class PuntoGeografico{
	private:
		int longitud ; 
		int latitud ;	
\end{lstlisting}
\end{enumerate}

\subsection{Escoger uno de los tipo rep y para éste establecer la función de abstracción}
Tras haber elegido la posibilidad 3, la \textbf{función de abstracción} se define como:\\
\begin{center}
	$ f_{A}(r) = $ $ " $r.longitud, r.latitud$ " $
\end{center}
Donde r es una instancia del objeto abstracto de tipo rep que sirve para representar el T.D.A. Punto Geográfico

\subsection{Invariante de la representación}
\begin{enumerate}[a)]
	\item $ r.longitud \in ['-180' - '180'] $ (condición de que la longitud debe encontrarse en el rango de valores especificado)
	\item $ r.latitud \in ['-90' - '90'] $ (condición de que la latitud debe encontrarse en el rango de valores especificado)
\end{enumerate}

\newpage

%----------------------------------------------------------------------------------------
%	Problema 4
%----------------------------------------------------------------------------------------
\section{Definir el T.D.A Ruta. Una ruta es una secuencia de puntos geográficos (ver ejercicio
	anterior)}
\subsection{Dar la especificación del tipo Ruta. Además establecer las operaciones que manejan
	a T.D.A}
T.D.A \textbf{Ruta}

\textbf{Especificación:} secuencia de puntos geográficos (p1,p2,...,pn).

\subsubsection{Operaciones:}
\begin{itemize}
	\item Constructores: constructor por defecto, constructor a partir de una Ruta determinada.
	\item Consulta: acceder a cualquiera de los puntos geográficos de la colección y al número total de éstos.
	\item Escritura y Lectura de una Ruta por un flujo de entrada y salida, respectivamente.
\end{itemize}

\subsection{Definir al menos dos tipo rep}
\begin{enumerate}[a)]
\item Posibilidad 1:
\begin{lstlisting}[style=cmas]
struct ruta{
	PuntoGeografico *punto ; 
	unsigned num_puntos ;		
\end{lstlisting}
\setlength{\parskip}{-4mm}
\item Posibilidad 2:
\begin{lstlisting}[style=cmas]
class Ruta{
	private:
		PuntoGeografico *punto ; 
		unsigned num_puntos ;	
\end{lstlisting}
\end{enumerate}

\subsection{Escoger uno de los tipo rep y para éste establecer la función de abstracción}
Tras haber elegido la posibilidad 2, la \textbf{función de abstracción} se define como:\\
\begin{center}
	$ f_{A}(r) = $ $ " $r.punto[0], r.punto[1],..., r.punto[r.num\_puntos - 1]$ " $
\end{center}
Donde r es una instancia del objeto abstracto de tipo rep que sirve para representar el T.D.A. Ruta

\subsection{Invariante de la representación}
\begin{center}
	$ (r.punto[i] \neq r.punto[j])\forall i,j $ $0 \leq i < j < r.num\_puntos $
\end{center}

\newpage

%----------------------------------------------------------------------------------------
%	Problema 5
%----------------------------------------------------------------------------------------
\section{Dar una especificación para la función que permite derivar un polinomio. Suponiendo que
	tenemos el TDA Polinomio, la cabecera de la función derivada sería asi:
	void Derivar(const Polinomio \& p\_origen, Polinomio \& p\_derivada);}

\begin{lstlisting}[style=cmas]
/**
* @brief Obtiene la derivada de un polinomio.
* @param p_origen: polinomio del que se va a obtener la derivada. 
* @param p_derivada: objeto Polinomio donde guarda la derivada. ES MODIFICADO
*/	
void Derivar(const Polinomio & p_origen, Polinomio & p_derivada);
\end{lstlisting}

\end{document}
