\input{preambuloSimple.tex}

%----------------------------------------------------------------------------------------
%	TÍTULO Y DATOS DEL ALUMNO
%----------------------------------------------------------------------------------------

\title{	
\normalfont \normalsize 
\textsc{\textbf{Estructura de Datos (2016-2017)} \\ Grupo C \\ Grado en Ingeniería Informática\\ Universidad de Granada} \\ [25pt] % Your university, school and/or department name(s)
\horrule{0.5pt} \\[0.4cm] % Thin top horizontal rule
\huge Relación 3: STL \\ % The assignment title
\horrule{2pt} \\[0.5cm] % Thick bottom horizontal rule
}

\author{Mario Rodríguez Ruiz} % Nombre y apellidos

\date{\normalsize\today} % Incluye la fecha actual

%----------------------------------------------------------------------------------------
% DOCUMENTO
%----------------------------------------------------------------------------------------

\begin{document}

\maketitle % Muestra el Título

\newpage %inserta un salto de página

\tableofcontents % para generar el índice de contenidos

\newpage

%----------------------------------------------------------------------------------------
%	Problema 1
%----------------------------------------------------------------------------------------
\section{Definir una función que permita invertir un objeto de tipo list. Los elementos que contiene la lista son
	enteros}
\begin{lstlisting}[style=cmas]
void Invertir(const list<int> & lsource, list<int>& ldestino)
{
	list<int>::const_reverse_iterator iter(lsource.rbegin()), iter_end(lsource.rend()) ;
	
	while (iter != iter_end){
		ldestino.push_back(*iter) ;
		++iter ;
	}
}		
\end{lstlisting}

\section{
	Definir una función que obtenga una cola de prioridad (priority\_queue) con la información de todos los
	alumnos, de manera que la prioridad se define de mayor a menor valor de selectividad.
}

\begin{lstlisting}[style=cmas]
bool operator <(const alumno &a1, const alumno &a2){
	if(a1.nota_selectividad < a2.nota_selectividad)
		return true ;
	return false ;
}

void ObtenerPrioridad (const list<alumno> & alumnos, priority_queue<alumno>  & pq){
	list<alumno>::const_iterator it;
	
	for(it = alumnos.begin(); it != alumnos.end(); ++it)
		pq.push(*it) ;
	
}		
\end{lstlisting}

\section{ Dada la clase list instanciada a enteros, crear una función que elimine los elementos pares de la
	lista. Para implementarla hacer uso de iteradores.
}

\begin{lstlisting}[style=cmas]
void EliminaPares(list<int> & lsource){
	vector<int> borrar ;
	list<int>::iterator it;
	
	for (it=lsource.begin(); it != lsource.end(); it++)
		if((*it)%2==0)
			borrar.push_back(*it) ;
	
	for(unsigned i = 0 ; i < borrar.size() ; i++)
		lsource.remove(borrar[i]) ;
}	
\end{lstlisting}

\section{Ejercicios 5 y 6}
\begin{lstlisting}[style=cmas]
class Alumnos{
	private:
		pair< string, set<string> > alumno;
	public:
		Alumnos(){}
		
		Alumnos(const pair<string, set<string> > a){
			alumno.first = a.first;
			alumno.second = a.second;
		}

		pair<string, set<string> > GetAlumno(){
			return alumno;
		}

		// Devuelve la cadena origen de un alumno.
		string GetAlumnoOrigen(){
			return alumno.first;
		}
		
		// Devuelve las asignaturas de un alumno.
		set<string> GetAlumnoDestino(){
			return alumno.second;
		}
		
		// Añade una asignatura a un alumno
		void AniadeAsignatura(string asig){
			alumno.second.insert(asig);
		}
};

class Matriculas{
	private:
		map<string, set<string> > matriculas; /**< Objeto de tipo map */
	
	public:
		Matriculas(){
			clear() ;
		}
		
		void clear(){
			matriculas.clear() ;
		}
		
		void Insertar(const string& dni,const string &cod_asig){
		pair<string, set<string> > d ;
		
		d.first = dni ;
		d.second.insert(cod_asig) ;
		Alumnos al(d) ;
		AniadeAlumno(al) ;
		}
		
		// Añade un alumno o asignaturas de éste
		void AniadeAlumno(Alumnos &p){
		Alumnos encontrado = GetAlumno(p.GetAlumnoOrigen());
		// Si el alumno no existe se crea uno nuevo
		// sino se inserta el nuevo código en el existente
		if(encontrado.GetAlumnoOrigen().size() == 0){
		pair<string, set<string> > e = p.GetAlumno();
		matriculas.insert(e);
		}
		else{
		set<string> enc_destino = p.GetAlumnoDestino();
		set<string>::iterator it_d;
		// Se recorre todas las nuevas asignaturas que
		// se van a ir insertando en el alumno original que ya existía.
		for ( it_d=enc_destino.begin(); it_d!=enc_destino.end(); ++it_d){
		matriculas[p.GetAlumnoOrigen()].insert(*it_d) ;
		}
		}
		}
		
		// Busca un alumno coincidente con la cadena pasada como parámetro
		Alumnos GetAlumno(const string origen){
		map<string, set<string> >::iterator it;
		it = matriculas.find(origen);
		
		if (it==matriculas.end())
		return Alumnos();
		else{
		Alumnos p(*it);
		return p;
		}
		}
		
		void Borrar(const string &dni,const string &cod_asig){
		Alumnos encontrado = GetAlumno(dni);
		// Si el alumno no existe no se hace nada
		if(encontrado.GetAlumnoOrigen().size() != 0){
		matriculas[encontrado.GetAlumnoOrigen()].erase(cod_asig) ;
		}
		}
		
		list<string> GetAsignatras(const string &dni){
		list <string> resultado;
		Alumnos encontrado = GetAlumno(dni);
		
		set<string> enc_destino = encontrado.GetAlumnoDestino();
		set<string>::iterator it;
		
		// Se recorren las asignaturas del alumno y se añaden a la nueva lista.
		for ( it=enc_destino.begin(); it!=enc_destino.end(); ++it){
		resultado.push_back(*it);
		}
		
		return resultado;
		}
		
		list<string> GetAlumnos(const string &cod_asig){
		list<string> resultado ;
		const_iterator it;		//Iterador para recorrer las matriculas
		set<string>::iterator ip; 			//Iterador para recorrer un set
		
		// Se recorre todas las matriculas.
		for (it = begin(); it!=end(); ++it){
		// Cada alumno de la matricula original
		Alumnos p(*it);
		set<string> destino = p.GetAlumnoDestino();
		// Se recorre el destino de cada Alumnos.
		for (ip = destino.begin(); ip!=destino.end(); ++ip){
		Alumnos buscar = GetAlumno(cod_asig);
		string dest_origen = buscar.GetAlumnoOrigen();
		// Si la alumno no existe se crea una nueva
		// sino se inserta la nueva traducción en la existente
		if(dest_origen.size() != 0){
		set<string> enc_destino = buscar.GetAlumnoDestino();
		set<string>::iterator it_d;
		// Se recorre todas las nuevas asignaturas que
		// se van a ir insertando en la lista.
		for ( it_d=enc_destino.begin(); it_d!=enc_destino.end(); ++it_d){
		resultado.push_back(*it_d) ;
		}
		}
		}
		}
		return resultado ;
		}
	
	class const_iterator{
	private:
	map<string, set<string> >::const_iterator it;
	public:
	const_iterator & operator++(){
	++it;
	return *this;
	}
	
	const_iterator & operator--(){
	--it;
	return *this;
	}
	
	Alumnos operator *(){
	Alumnos p(*it);
	return p;
	}
	
	bool operator ==(const const_iterator &i){
	return i.it==it;
	}
	
	bool operator !=(const const_iterator &i){
	return i.it!=it;
	}
	
	friend class Matriculas;
	};
	
	
	const_iterator begin() const {
	const_iterator i;
	i.it=matriculas.begin();
	return i;
	}
	
	
	const_iterator end() const {
	const_iterator i;
	i.it=matriculas.end();
	return i;
	}
	
	
	const_iterator buscar(string f) {
	const_iterator i;
	i.it=matriculas.find(f);
	return i;
	}

};
\end{lstlisting}

%\subsection{Definir al menos dos tipo rep}
%\begin{enumerate}[a)]
%\item Posibilidad 1:
%\begin{lstlisting}[style=cmas]
%class ServidorRed{
%	private:
%		int d1, d2, d3, d4 ;		
%\end{lstlisting}

\end{document}
