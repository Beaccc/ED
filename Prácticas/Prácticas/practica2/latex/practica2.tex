%%%%%%%%%%%%%%%%%%%%%%%%%%%%%%%%%%%%%%%%%
% Short Sectioned Assignment LaTeX Template Version 1.0 (5/5/12)
% This template has been downloaded from: http://www.LaTeXTemplates.com
% Original author:  Frits Wenneker (http://www.howtotex.com)
% License: CC BY-NC-SA 3.0 (http://creativecommons.org/licenses/by-nc-sa/3.0/)
%%%%%%%%%%%%%%%%%%%%%%%%%%%%%%%%%%%%%%%%%

%----------------------------------------------------------------------------------------
%	PACKAGES AND OTHER DOCUMENT CONFIGURATIONS
%----------------------------------------------------------------------------------------

\documentclass[paper=a4, fontsize=11pt]{scrartcl} % A4 paper and 11pt font size

% ---- Entrada y salida de texto -----

\usepackage[T1]{fontenc} % Use 8-bit encoding that has 256 glyphs
\usepackage[utf8]{inputenc}
%\usepackage{fourier} % Use the Adobe Utopia font for the document - comment this line to return to the LaTeX default

% ---- Idioma --------

\usepackage[spanish, es-tabla]{babel} % Selecciona el español para palabras introducidas automáticamente, p.ej. "septiembre" en la fecha y especifica que se use la palabra Tabla en vez de Cuadro

% ---- Otros paquetes ----

\usepackage{url} % ,href} %para incluir URLs e hipervínculos dentro del texto (aunque hay que instalar href)
\usepackage{amsmath,amsfonts,amsthm} % Math packages
%\usepackage{graphics,graphicx, floatrow} %para incluir imágenes y notas en las imágenes
\usepackage{graphics,graphicx, float} %para incluir imágenes y colocarlas

% Para hacer tablas comlejas
%\usepackage{multirow}
%\usepackage{threeparttable}

%\usepackage{sectsty} % Allows customizing section commands
%\allsectionsfont{\centering \normalfont\scshape} % Make all sections centered, the default font and small caps

\usepackage{fancyhdr} % Custom headers and footers

\pagestyle{fancyplain} % Makes all pages in the document conform to the custom headers and footers
\fancyhead{} % No page header - if you want one, create it in the same way as the footers below
\fancyfoot[L]{} % Empty left footer
\fancyfoot[C]{} % Empty center footer
\fancyfoot[R]{\thepage} % Page numbering for right footer
\renewcommand{\headrulewidth}{0pt} % Remove header underlines
\renewcommand{\footrulewidth}{0pt} % Remove footer underlines
\setlength{\headheight}{13.6pt} % Customize the height of the header

\numberwithin{equation}{section} % Number equations within sections (i.e. 1.1, 1.2, 2.1, 2.2 instead of 1, 2, 3, 4)
\numberwithin{figure}{section} % Number figures within sections (i.e. 1.1, 1.2, 2.1, 2.2 instead of 1, 2, 3, 4)
\numberwithin{table}{section} % Number tables within sections (i.e. 1.1, 1.2, 2.1, 2.2 instead of 1, 2, 3, 4)

\setlength\parindent{0pt} % Removes all indentation from paragraphs - comment this line for an assignment with lots of text

\newcommand{\horrule}[1]{\rule{\linewidth}{#1}} % Create horizontal rule command with 1 argument of height
\usepackage[breaklinks=true]{hyperref}

\usepackage[dvipsnames]{xcolor}
\usepackage{amssymb}
\usepackage{color}
\usepackage{listings}
\usepackage{upgreek} % para poner letras griegas sin cursiva
\usepackage{cancel} % para tachar
\usepackage{mathdots} % para el comando \iddots
\usepackage{mathrsfs} % para formato de letra
\usepackage{stackrel} % para el comando \stackbin
\lstset{ %
language=C++,                % elegir el lenguaje del código
stringstyle=\color{blue}\ttfamily,,
basicstyle=\normalsize\ttfamily,       % el tamaño del font a usar para el código
numbers=left,                   % dónde poner los números de línea 
numberstyle=\footnotesize,      % tamaño de font usados para los números de línea 
stepnumber=1,                   % el paso de numeración
numbersep=5pt,                  % distancia del numero de línea y la línea
backgroundcolor=\color{white},  % color de fondo, para usarlo hay que agregar  \usepackage{color}
showspaces=false,               % mostrar espacios en blanco ?
showstringspaces=false,         % subrayar espacios con cadenas?   
 showtabs=false,                 % mostrar taba usando cadenas? 
frame=single,           			% enmarcar el código?  
tabsize=2,          				% sets default tabsize to 2 spaces?
keywordstyle=\color{MidnightBlue}\ttfamily\bfseries,
commentstyle=\color{OliveGreen}\ttfamily,
morecomment=[l][\color{OliveGreen}]{\#},
captionpos=b,           % sets the caption-position to bottom?
breaklines=true,        % sets automatic line breaking?
breakatwhitespace=false,    % sets if automatic breaks should only happen at whitespace ?
title=\lstname,
escapeinside={\%*}{*)}          % if you want to add a comment within your code
}

\lstset{literate=
  {á}{{\'a}}1 {é}{{\'e}}1 {í}{{\'i}}1 {ó}{{\'o}}1 {ú}{{\'u}}1
  {Á}{{\'A}}1 {É}{{\'E}}1 {Í}{{\'I}}1 {Ó}{{\'O}}1 {Ú}{{\'U}}1
  {à}{{\`a}}1 {è}{{\`e}}1 {ì}{{\`i}}1 {ò}{{\`o}}1 {ù}{{\`u}}1
  {À}{{\`A}}1 {È}{{\'E}}1 {Ì}{{\`I}}1 {Ò}{{\`O}}1 {Ù}{{\`U}}1
  {ä}{{\"a}}1 {ë}{{\"e}}1 {ï}{{\"i}}1 {ö}{{\"o}}1 {ü}{{\"u}}1
  {Ä}{{\"A}}1 {Ë}{{\"E}}1 {Ï}{{\"I}}1 {Ö}{{\"O}}1 {Ü}{{\"U}}1
  {â}{{\^a}}1 {ê}{{\^e}}1 {î}{{\^i}}1 {ô}{{\^o}}1 {û}{{\^u}}1
  {Â}{{\^A}}1 {Ê}{{\^E}}1 {Î}{{\^I}}1 {Ô}{{\^O}}1 {Û}{{\^U}}1
  {œ}{{\oe}}1 {Œ}{{\OE}}1 {æ}{{\ae}}1 {Æ}{{\AE}}1 {ß}{{\ss}}1
  {ű}{{\H{u}}}1 {Ű}{{\H{U}}}1 {ő}{{\H{o}}}1 {Ő}{{\H{O}}}1
  {ç}{{\c c}}1 {Ç}{{\c C}}1 {ø}{{\o}}1 {å}{{\r a}}1 {Å}{{\r A}}1
  {€}{{\EUR}}1 {£}{{\pounds}}1
  {ñ}{{\~n}}1
}



\hypersetup{
    colorlinks=true,
    linkcolor=black,
    filecolor=magenta,      
    urlcolor=blue,
    pdftitle={ED: Práctica 3 - Mario Rodríguez Ruiz},
    bookmarks=true,
}



%----------------------------------------------------------------------------------------
%	TÍTULO Y DATOS DEL ALUMNO
%----------------------------------------------------------------------------------------

\title{	
\normalfont \normalsize 
\textsc{\textbf{Estructura de Datos (2016-2017)} \\ Subgrupo C2 \\ Grado en Ingeniería Informática\\ Universidad de Granada} \\ [25pt] % Your university, school and/or department name(s)
\horrule{0.5pt} \\[0.4cm] % Thin top horizontal rule
\huge Práctica 2: Documentación del Software \\ % The assignment title
\horrule{2pt} \\[0.5cm] % Thick bottom horizontal rule
}

\author{Mario Rodríguez Ruiz} % Nombre y apellidos

\date{\normalsize\today} % Incluye la fecha actual

%----------------------------------------------------------------------------------------
% DOCUMENTO
%----------------------------------------------------------------------------------------

\begin{document}

\maketitle % Muestra el Título

\newpage %inserta un salto de página

\tableofcontents % para generar el índice de contenidos

\newpage

%----------------------------------------------------------------------------------------
%	Cuestión 1
%----------------------------------------------------------------------------------------

\section{Especificación}
\textbf{\textit{Dar la especificación. Establecer una definición y el conjunto de operaciones básicas.}}

\subsection{T.D.A. Frase}
Almacena una frase en un idioma de origen junto con sus distintas traducciones a otro idioma, 
ya que una frase puede tener significados distintos.
\\

\textbf{\textit{Operaciones}:}
\begin{itemize}
	\item Constructores: constructor por defecto, constructor de copia, constructor con un tamaño determinado.
	\item Consulta: acceder a la frase origen, a las traducciones y al número de ellas.
	\item Modificadores: de la frase de origen y de las traducciones.	
	\item Operadores: Escritura y Lectura de una frase por un flujo de entrada y salida, respectivamente; comparar alfabéticamente dos frases.
\end{itemize}

\subsection{T.D.A. ConjuntoFrases}
Almacena un grupo de frases junto con sus traducciones ordenadas alfabéticamente en función de la frase origen.
Para cada frase en un idioma origen se asocian un conjunto de frases, en las que se traduce la frase origen en el idioma destino. 

Entradas posibles en un traductor español-inglés / inglés-español:
\begin{itemize}
	\item Give a dog a bad name and hang it;Hazte fama y échate a dormir;Un perro maté y mataperros me llamaron
	\item Lo tienes claro;You’ve got another thing coming
\end{itemize}


\textbf{\textit{Operaciones}:}
\begin{itemize}
	\item Constructores: constructor por defecto, constructor de copia, constructor con un tamaño determinado.
	\item Consulta: acceder al número de frases almacenadas, a una frase específica, a un conjunto de traducciones y a las frases que contengan una cadena.	
	\item Operadores: Escritura y lectura de un conjunto de frases por un flujo de entrada y salida, respectivamente.
\end{itemize}

\newpage

%----------------------------------------------------------------------------------------
%	Cuestión 2
%----------------------------------------------------------------------------------------

\section{Diferentes estructuras tipo rep}
\textbf{\textit{Determinar diferentes estructuras de datos para tipo rep.}}

\subsection{T.D.A. Frase}
\begin{lstlisting}
	class Frase{
		private:
			string origen ;         
			string *traducciones ;  
			int num_trad ;  
				
	class Frase{
		private:        
			string origen ;         
			string **traducciones ; 
			int num_orig ; 
			int num_trad ;
			
	struct Frase{
		string origen;
		vector<string> traducciones;
	};
\end{lstlisting}

\subsection{T.D.A. ConjuntoFrases}
\begin{lstlisting}
	class ConjuntoFrases{
		private:
			Frase *frases ;     
			int num_frases ;
	
	class ConjuntoFrases {
		private:
			vector<Frase> frases; 	
\end{lstlisting}

\newpage

%----------------------------------------------------------------------------------------
%	Cuestión 3
%----------------------------------------------------------------------------------------

\section{Estructuras de datos para representar el tipo rep}
\textbf{\textit{Escoger una de las estructuras de datos para representar el tipo rep}}

\subsection{T.D.A. Frase}
\begin{lstlisting}
	class Frase{
		private:
			string origen ;         
			string *traducciones ;  
			int num_trad ;  
\end{lstlisting}

\subsection{T.D.A. ConjuntoFrases}
\begin{lstlisting}
	class ConjuntoFrases{
		private:
			Frase *frases ;     
			int num_frases ;
\end{lstlisting}

%----------------------------------------------------------------------------------------
%	Cuestión 4
%----------------------------------------------------------------------------------------

\section{Invariante de la representación y función de abstracción}
\textbf{\textit{ Para la estructura de datos del tipo rep
		establecer cual es el invariante de la
		representación y función de abstracción.}}

\subsection{Invariante de la representación}
\textbf{T.D.A. Frase}
\begin{center}
	$\forall i \ r.frases[i].traducciones.size() > 0$
\end{center} 
Condición de que cada frase origen tiene una frase destino.
\\

\textbf{T.D.A. ConjuntoFrases}
\begin{center}
	$\forall i,j \ \textup{tales que } i < j \rightarrow r.frases[i].origen < r.frases[j].origen, 0 \leq i,j < n$
\end{center} 
Expresa que el conjunto está ordenado por la frase origen.

\newpage

\subsection{Función de abstracción}
\textbf{T.D.A. Frase}
\begin{displaymath}
f_{A}(r) =\{(r.origen;r.traducciones[0], ... , r.traducciones[n-1])\}
\end{displaymath}

\textbf{T.D.A. ConjuntoFrases}

\begin{displaymath}
f_{A}(r) =\{(r.frases[0].origen;r.frases[0].traducciones[0],...,
\end{displaymath}
\begin{displaymath}
 frases[0].traducciones[frases[0].traducciones.size()-1]),
\end{displaymath}
\begin{displaymath}
\ldots
\end{displaymath}
\begin{displaymath}
(r.frases[r.frases.size-1].origen;r.frases[r.frases.size()-1].traducciones[0],...,
\end{displaymath}
\begin{displaymath}
[r.frases.size()-1].traducciones[r.frases[0].traducciones.size()-1])\}
\end{displaymath}
\end{document}
